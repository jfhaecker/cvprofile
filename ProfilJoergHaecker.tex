\documentclass[11pt, a4paper, sans]{moderncv}
\moderncvstyle{classic}
\moderncvcolor{blue}
\usepackage[scale=0.75]{geometry}
\usepackage{ngerman}
\usepackage[utf8]{inputenc}
\name{Jörg}{Häcker}
\address{Ebertstr. 33}{76135 Karlsruhe}{Deutschland}
\phone[mobile] {01712812016}
\email{joerghaecker@gmx.net}
\photo{joerg.png}
\begin{document}
\maketitle

\section{Über mich}
\cvline{}{
In meiner Arbeit als Scrum Master liegt mir die Weiterentwicklung des Teams besonders am Herzen. Eine auf Vertrauen und Wertschätzung basierende Zusammenarbeit innerhalb des gesamten Scrum-Teams sehe ich als wichtigen Faktor für den Erfolg von Scrum. Ich verfüge über einen  technischen Background und langjährige Erfahrung als Führungskraft in der Softwareentwicklung.
}

\section{Erfahrung}

\cventry{11/17--08/18}{Scrum Master}{Deutsche Bahn AG}{Frankfurt a.M.}{}{
Entwicklung einer neuen, zentralen Plattform für Reisendeninformationen bei der Deutschen Bahn AG
\newline{}\newline{}
Leistungen
\begin{itemize}
\item Scrum Master für mehrere Entwicklungsteams in einem auf dem \emph{Nexus}-Framework basierenden Skalierungsansatz.
\item Moderation von Team-internen und Team-übergreifenden Scrum-Events.
\item Coaching des Product Owner im Hinblick auf Backlog-Management, User Storys und Terminplanung.
\item Moderation des \emph{Nexus}-Daily bestehend aus Abgesandten der Teams, um einen gemeinsamen Plan für die nächsten 24h zu entwickeln.
\item Moderation und Vorbereitung des Product Owner Planning zur Definition und Planung von gemeinsamen projektweiten Zielen.
\item Coaching der Organisation bezüglich der Bedeutung des \emph{Done}-Inkrements.
\item Operative und strategische Abstimmung mit anderen Scrum Mastern und dem Management.
%\begin{itemize}
%\item Paper jam
%item Software issues:
%\begin{itemize}
%\item Word not sending the correct data to printer
%\item Windows trying to print in letter format
%\end{itemize}
%\item Coffee spilled inside printer
%\end{itemize}
%\item Broke the office record for number of kitten pictures in cubicle
\end{itemize}}


\cventry{06/15--03/16}{Scrum Master}{Daimler TSS}{Berlin}{}{
 Weiterentwicklung und Stabilisierung des Fahrzeugmanagementsystem zur Verwaltung von Serviceverträgen für Nutzfahrzeuge
 \newline{}\newline{}
Leistungen
\begin{itemize}
\item Scrum Master für ein Entwicklungsteam bestehend aus internen und externen Mitarbeitern.
\item Coaching des Teams bezüglich der Erstellung eines \emph{Done}-Inkrements.
\item Unterstützung von internen IT-Stakeholdern und des Product Owner bei der Einlastung von zentralen IT-Themen.
\item Abstimmung mit anderen Scrum Master und Urlaubsvertretung beim Schwesternteam.
\item Unterstützung des Management bei der Personalauswahl.
\end{itemize}}


\cventry{04/12--04/14}{Head of Home Access Processes}{1\&1 Internet AG}{Karlsruhe}{Festanstellung}{
Fachliche, disziplinarische und organisatorische Führungskraft eines Softwareentwicklungsteams im JEE Umfeld. 
\newline{}\newline{}
Leistungen
\begin{itemize}
\item Verantwortlich für Entwicklung und Betrieb der steuernden Geschäftsprozesse für die Auftragsbearbeitung der Produkte für standortgebundene Internetanschlüsse (DSL) der 1\&1. 
\item Durchführung von Bewerbungsgesprächen für Festangestellte und Auswahl geeigneter externer Mitarbeiter.
\item Personalentwicklung, Beurteilung und Coaching von Mitarbeitern.
\item Weiterentwicklung des Teams zur Unterstützung der strategischen Bereichsziele.
\item Einführung agiler Softwareentwicklungsmethoden.
\item Ressourcenplanung für alle Softwareentwicklungsprojekte des Teams und Projektmanagement für ausgewählte Technikprojekte.
\item Sicherstellung der Termintreue der im Team umgesetzten Roadmap Projekte.
\end{itemize}}

\cventry{02/08--04/12}{Head of DSL Services} {1\&1 Internet AG}{Karlsruhe}{Festanstellung}{
Fachliche, disziplinarische und organisatorische Führungskraft eines Softwareentwicklungsteams im JEE Umfeld
\newline{}\newline{}
Leistungen
\begin{itemize}
\item Verantwortlich für Entwicklung und Betrieb des 1\&1 VoIP Rufnummernverwaltungssystems und des 1\&1 AutoConfigurationServer.
\item Durchführung von Bewerbungsgesprächen für Festangestellte und Auswahl geeigneter externer Mitarbeiter.
\item Personalentwicklung, Beurteilung und Coaching von Mitarbeitern.
\item Weiterentwicklung des Teams zur Unterstützung der strategischen Bereichsziele.
\item Einführung agiler Softwareentwicklungsmethoden.
\item Ressourcenplanung für alle Softwareentwicklungsprojekte des Teams und Projektmanagement für ausgewählte Technikprojekte.
\item Sicherstellung der Termintreue der im Team umgesetzten Roadmap Projekte.
\end{itemize}}

\cventry{04/05--02/08}{Teamleiter Pustefix} {1\&1 Internet AG}{Karlsruhe}{Festanstellung}{
Fachliche, disziplinarische und organisatorische Führungskraft eines Softwareentwicklungsteams im JEE Umfeld
\newline{}\newline{}
Tätigkeiten
\begin{itemize}
\item Verantwortlich für die Weiterentwicklung des hausinternen Web-Frameworks Pustefix und zugehöriger Werkzeuge.
\end{itemize}}

\cventry{06/02--04/05}{Softwareentwickler} {1\&1 Internet AG}{Karlsruhe}{Festanstellung} {
Durchführung von Projekten im Bereich Webanwendungen für die Bestell- und Endkundenkonfigurationssysteme der 1und1 sowie Weiterentwicklung des hausinternen Webapplikationsframeworks Pustefix.
}

\cventry{1995-2002} {Studium Informatik} {FH Karlsruhe} {}{} {
Studium und Abschluss als Diplom-Informatiker (FH).
} 

\section{Weiterbildung}
\cventry{2017}{Professional Scrum Master II}{scrum.org}{}{}{}
\cventry{2016}{Kanban Management Professional I}{it-agile}{Hamburg}{} {}
\cventry{2015}{Certified LeSS Practitioner}{Craig Larman}{Berlin}{} {}
\cventry{2015}{Certified SAFe Agilist}{wibas GmbH}{Darmstadt}{}{}
\cventry{2015}{Professional Scrum Product Owner I}{scrum.org}{}{}{}
\cventry{2014}{Professional Scrum Master I}{scrum.org}{}{}{}
\cventry{2011}{Theorie und Anwendung von Persönlichkeitsmodellen}{Dart Consulting}{Karlsruhe}{}{}
\cventry{2008--2014}{Leadership Development Program 3}{1\&1 Internet AG}{Karlsruhe}{}{}

\section{Kenntnisse}

\cvitem{Agile}{}
\cvitem{}{%
  \begin{tabular}{p{5cm}l}
  Scrum Master        & ++++ \\
  Product Owner       & +++  \\
  Kanban              & +++   \\
  \end{tabular}%
}

\cvitem{Management}{}
\cvitem{}{%
  \begin{tabular}{p{5cm}l}
  Teamleitung                  & ++++ \\
  Personalauswahl              & ++++ \\
  Personalentwicklung          & ++++  \\
  Resourcen- u. Terminplanung  & ++++ \\
  Strategieentwicklung         & +++   \\
  \end{tabular}%
}

\cvitem{Technik}{}
\cvitem{}{%
  \begin{tabular}{p{5cm}l}
  JEE Technologien             & +++ \\
  DevOps                       & +++ \\
  Continous Integration        & +++  \\
  BPMN                         & +++  \\
  Datenbanken                  & +++ \\
  JIRA                         & +++   \\
  Hochlastsysteme              & +++ \\
  \end{tabular}%
}
\cvitem{    }{}

\cvitem{+++++}{Exzellente Kenntnisse. Löst aktiv schwierigste Fragestellungen.}
\cvitem{++++}{Fortgeschrittene Kenntnisse. Löst aktiv schwierige Fragestellungen.}
\cvitem{+++}{Gute Mittelstufe. Setzt Aufgaben im Fachgebiet selbstständig um.}
\cvitem{++}{Mittelstufe. Setzt Routineaufgaben selbstständig ohne Anleitung um.}
\cvitem{+}{Grundlagen. Kann einfache Routineaufgaben unter Anleitung umsetzen.}
\cvitem{o}{Einstieg. Kann Anwendung und Hintergründe grob erläutern.}



\end{document}
