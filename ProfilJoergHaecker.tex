
\documentclass[11pt, a4paper, sans]{moderncv}
\moderncvstyle{classic}
\moderncvcolor{blue}
\usepackage[scale=0.75]{geometry}
\usepackage{ngerman}
\usepackage[utf8]{inputenc}
\name{Jörg}{Häcker}
\address{Ebertstr. 33}{76135 Karlsruhe}{Deutschland}
\phone[mobile] {01712812016}
\email{joerghaecker@gmx.net}
\photo{joerg.png}
\begin{document}
\maketitle

\section{Warum Agile?}
\cvline{}{
Kundenbedürfnisse befriedigen - Das ist das Ziel. 
}
\cvline{}{
Selbstorganisierende Teams erhalten Feedback, indem sie funktionierende Software möglichst früh und  kontinuierlich liefern - Das ist der Weg.
}
\cvline{}{
In meiner Laufbahn als Teamleiter in der Softwareentwicklung ab 2005 wurde mir schon früh bewusst, dass Command \& Control, festzementierte Pläne sowie Einzelkämpfertum in der Entwicklung von komplexen Produkten der falsche Weg ist. Seitdem habe ich zahlreiche agile Pflänzchen keimen, wachsen, gedeihen aber leider auch stagnieren sehen. Durch die Einführung und Verfeinerung agiler Methoden unterstütze ich Unternehmen in einer komplexer werdenden Welt erfolgreich zu sein.
}
\cvline{}{
In meiner Arbeit als Scrum Master betone ich besonders das Wertesystem von Scrum und die Bedeutung eines ``Done``-Inkrements, das auch wirklich ``Done`` ist. 
}


\section{Das zeichnet mich aus}
\cvline{}{
\begin{itemize} 
\item Ausgeprägte Hands-on-Mentalität
\item Kompromisslose Ausrichtung der Führung am Interesse des Teams
\item Sehr gute Kenntnisse der gängigen agilen Frameworks in Theorie und Praxis
\item Zielgerichtete Moderation von Meetings
\item Aktives Coaching der Teams und der Organisation zu agilen Werten
\item Respektvolle und wertschätzende Kommunikation mit allen Beteiligten
\item Tiefes Verständnis der Arbeitsweise in Entwicklung und Betrieb von Softwareprodukten in großen Unternehmen
\end{itemize}
}

\section{Erfahrung}



\cventry{12/19-11/20}{Scrum Master}{Destatis - Statistisches Bundesamt}{Wiesbaden}{}{
Zensus2021 - Statistische Erhebung, wie viele Menschen in Deutschland leben, wie sie wohnen und arbeiten.
\newline{}\newline{}
Leistungen
\begin{itemize}
\item Scrum Master im Teilprojekt Gebäude- und Wohnungszählung mit 3-4 Entwickungsteams in einem skalierten Scrum-Umfeld.
\item Organisation und Outcome-orientierte Moderation sämtlicher Scrum-Meetings.
\item Coaching des Product-Owner Teams im Hinblick auf Backlog-Management, User Stories und Terminplanung.
\item Optimierung der Zusammenarbeit zwischen Enticklungsteams, Anforderungsanalyse, Fachbereich, Architektur und Qualitätssicherung.
\item Einführung und Erhaltung kontinuierlicher Verbesserungsprozesse in Abstimmung mit der Projektleitung.
\item Unterstützung beim Defect-Managment mittels JIRA-Werkzeugen. 
\item Anpassung der VorOrt-Arbeitsweise auf Remote-Arbeitsweise.
\end{itemize}} 

\cventry{2017--2018}{Scrum Master}{Deutsche Bahn AG}{Frankfurt a.M.}{}{
Entwicklung einer neuen, zentralen Plattform für Reisendeninformationen bei der Deutschen Bahn AG
\newline{}\newline{}
Leistungen
\begin{itemize}
\item Scrum Master für mehrere Entwicklungsteams in einem skalierten Scrum-Umfeld (50 MA+).
\item Moderation von Team-internen und Team-übergreifenden Scrum-Events.
\item Coaching des Product Owner im Hinblick auf Backlog-Management, User Storys und Terminplanung.
\item Moderation des Nexus Daily bestehend aus Abgesandten aller Teams, um einen gemeinsamen Plan für die nächsten 24h zu entwickeln.
\item Moderation und Vorbereitung des Product Owner Planning zur Definition und Planung von gemeinsamen projektweiten Zielen.
\item Coaching der Organisation bezüglich der Bedeutung des ``Done``-Inkrements.
\item Operative und strategische Abstimmung mit anderen Scrum Mastern und dem Management.
%\begin{itemize}
%\item Paper jam
%item Software issues:
%\begin{itemize}
%\item Word not sending the correct data to printer
%\item Windows trying to print in letter format
%\end{itemize}
%\item Coffee spilled inside printer
%\end{itemize}
%\item Broke the office record for number of kitten pictures in cubicle
\end{itemize}}





\cventry{2015--2016}{Scrum Master}{Daimler TSS}{Berlin}{}{
 Weiterentwicklung und Stabilisierung des Fahrzeugmanagementsystem zur Verwaltung von Serviceverträgen für Nutzfahrzeuge
 \newline{}\newline{}
Leistungen
\begin{itemize}
\item Scrum Master für ein Entwicklungsteam bestehend aus internen und externen Mitarbeitern.
\item Coaching des Teams bezüglich der Erstellung eines ``Done``-Inkrements.
\item Unterstützung von internen IT-Stakeholdern und des Product Owner bei der Einlastung von zentralen IT-Themen.
\item Abstimmung mit anderen Scrum Master und Urlaubsvertretung beim Schwesternteam.
\item Unterstützung des Management bei der Personalauswahl.
\end{itemize}}




\cventry{2012--2014}{Head of Home Access Processes}{1\&1 Internet AG}{Karlsruhe}{Festanstellung}{
Fachliche, disziplinarische und organisatorische Führungskraft eines Softwareentwicklungsteams im JEE Umfeld. 
\newline{}\newline{}
Leistungen
\begin{itemize}
\item Verantwortlich für Entwicklung und Betrieb der steuernden Geschäftsprozesse für die Auftragsbearbeitung der Produkte für standortgebundene Internetanschlüsse (DSL) der 1\&1. 
\item Durchführung von Bewerbungsgesprächen für Festangestellte und Auswahl geeigneter externer Mitarbeiter.
\item Personalentwicklung, Beurteilung und Coaching von Mitarbeitern.
\item Weiterentwicklung des Teams zur Unterstützung der strategischen Bereichsziele.
\item Einführung agiler Softwareentwicklungsmethoden.
\item Ressourcenplanung für alle Softwareentwicklungsprojekte des Teams und Projektmanagement für ausgewählte Technikprojekte.
\item Sicherstellung der Termintreue der im Team umgesetzten Roadmap Projekte.
\end{itemize}}

\cventry{2008--2012}{Head of DSL Services} {1\&1 Internet AG}{Karlsruhe}{Festanstellung}{
Fachliche, disziplinarische und organisatorische Führungskraft eines Softwareentwicklungsteams im JEE Umfeld
\newline{}\newline{}
Leistungen
\begin{itemize}
\item Verantwortlich für Entwicklung und Betrieb des 1\&1 VoIP Rufnummernverwaltungssystems und des 1\&1 AutoConfigurationServer.
\item Durchführung von Bewerbungsgesprächen für Festangestellte und Auswahl geeigneter externer Mitarbeiter.
\item Personalentwicklung, Beurteilung und Coaching von Mitarbeitern.
\item Weiterentwicklung des Teams zur Unterstützung der strategischen Bereichsziele.
\item Einführung agiler Softwareentwicklungsmethoden.
\item Ressourcenplanung für alle Softwareentwicklungsprojekte des Teams und Projektmanagement für ausgewählte Technikprojekte.
\item Sicherstellung der Termintreue der im Team umgesetzten Roadmap Projekte.
\end{itemize}}

\cventry{2005--2008}{Teamleiter Pustefix} {1\&1 Internet AG}{Karlsruhe}{Festanstellung}{
Fachliche, disziplinarische und organisatorische Führungskraft eines Softwareentwicklungsteams im JEE Umfeld
\newline{}\newline{}
Tätigkeiten
\begin{itemize}
\item Verantwortlich für die Weiterentwicklung des hausinternen Web-Frameworks Pustefix und zugehöriger Werkzeuge.
\end{itemize}}

\cventry{2002--2005}{Softwareentwickler} {1\&1 Internet AG}{Karlsruhe}{Festanstellung} {
Durchführung von Projekten im Bereich Webanwendungen für die Bestell- und Endkundenkonfigurationssysteme der 1und1 sowie Weiterentwicklung des hausinternen Webapplikationsframeworks Pustefix.
}

\cventry{1995-2002} {Studium Informatik} {FH Karlsruhe} {}{} {
Studium und Abschluss als Diplom-Informatiker (FH).
} 

\section{Weiterbildung}
\cventry{2018}{Traning ``Professional Scrum Master II''}{andrena objects ag}{Karlsruhe}{}{}
\cventry{2017}{Professional Scrum Master II}{scrum.org}{}{}{}
\cventry{2016}{Kanban Management Professional I}{it-agile}{Hamburg}{} {}
\cventry{2015}{Certified LeSS Practitioner}{Craig Larman}{Berlin}{} {}
\cventry{2015}{Certified SAFe Agilist}{wibas GmbH}{Darmstadt}{}{}
\cventry{2015}{Professional Scrum Product Owner I}{scrum.org}{}{}{}
\cventry{2014}{Professional Scrum Master I}{scrum.org}{}{}{}
\cventry{2011}{Theorie und Anwendung von Persönlichkeitsmodellen}{Dart Consulting}{Karlsruhe}{}{}
\cventry{2008--2014}{Leadership Development Program 3}{1\&1 Internet AG}{Karlsruhe}{}{}

\section{Kenntnisse}

\cvitem{Agile}{}
\cvitem{}{%
  \begin{tabular}{p{5cm}l}
  Scrum               & ++++ \\
  Nexus               & ++++  \\
  SaFe                & +++  \\
  LeSS                & +++  \\
  Kanban              & +++   \\
  \end{tabular}%
}

\cvitem{Management}{}
\cvitem{}{%
  \begin{tabular}{p{5cm}l}
  Teamleitung                  & ++++ \\
  Personalauswahl              & ++++ \\
  Personalentwicklung          & ++++  \\
  Resourcen- u. Terminplanung  & ++++ \\
  Strategieentwicklung         & +++   \\
  \end{tabular}%
}

\cvitem{Technik}{}
\cvitem{}{%
  \begin{tabular}{p{5cm}l}
  Linux                        & ++++ \\
  DevOps                       & +++ \\
  Continous Integration        & +++  \\
  TDD                          & +++  \\
  Cloud                        & +++ \\
  JIRA                         & ++++   \\
  Python                       & +++ \\
  JAVA                         & +++ \\
  \end{tabular}%
}
\cvitem{    }{}

\cvitem{+++++}{Exzellente Kenntnisse. Löst aktiv schwierigste Fragestellungen.}
\cvitem{++++}{Fortgeschrittene Kenntnisse. Löst aktiv schwierige Fragestellungen.}
\cvitem{+++}{Gute Mittelstufe. Setzt Aufgaben im Fachgebiet selbstständig um.}
\cvitem{++}{Mittelstufe. Setzt Routineaufgaben selbstständig ohne Anleitung um.}
\cvitem{+}{Grundlagen. Kann einfache Routineaufgaben unter Anleitung umsetzen.}
\cvitem{o}{Einstieg. Kann Anwendung und Hintergründe grob erläutern.}



\end{document}
