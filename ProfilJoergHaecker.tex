\documentclass[11pt, a4paper, sans]{moderncv}
\moderncvstyle{classic}
\moderncvcolor{blue}
\usepackage[scale=0.75]{geometry}
\usepackage{ngerman}
\usepackage[utf8]{inputenc}
\name{Jörg}{Häcker}
\address{Ebertstr. 33}{76135 Karlsruhe}{Deutschland}
\phone[mobile] {01712812016}
\email{joerghaecker@gmx.net}
\photo{joerg.png}
\begin{document}
\maketitle

\section{Über mich}
\cvline{}{
In meiner Arbeit als Scrum Master liegt mir die Weiterentwicklung des Teams besonders am Herzen. Eine auf Vertrauen und Wertschätzung basierende Zusammenarbeit innerhalb des gesamten Scrum-Teams sehe ich als wichtigen Faktor für den Erfolg von Scrum. Ich verfüge über einen  technischen Background und langjährige Erfahrung als Führungskraft in der Softwareentwicklung.
}

\section{Erfahrung}

\cventry{11/17--08/18}{Scrum Master}{Deutsche Bahn AG}{Frankfurt a.M.}{}{
Scrum Master in einem auf dem \emph{Nexus}-Framework basierenden Skalierungsansatz. Ziel des Projekts ist die  Entwicklung einer neuen Plattform für Reisendeninformationen als \emph{Single Point of Truth}.
}

\cventry{06/15--03/16}{Scrum Master}{Daimler TSS}{Berlin}{}{
Scrum Master für ein Entwicklungsteam mit Fokus auf einer Scrum-Implementierung basierend auf dem \emph{Scrum Guide}.
Projektauftrag war die Weiterentwicklung und Stabilisierung des Fahrzeugmanagementsystem zur Verwaltung von Serviceverträgen für Nutzfahrzeuge.
}


\cventry{04/12--04/14}{Head of Home Access Processes}{1\&1 Internet AG}{Karlsruhe}{Festanstellung}{
Fachliche, disziplinarische und organisatorische Führungskraft eines Softwareentwicklungsteams im JEE Umfeld. Verantwortlich für Entwicklung und Betrieb der steuernden Geschäftsprozesse für die Auftragsbearbeitung der Produkte für standortgebundene Internetanschlüsse (DSL) der 1\&1. }

\cventry{02/08--04/12}{Head of DSL Services} {1\&1 Internet AG}{Karlsruhe}{Festanstellung}{
Fachliche, disziplinarische und organisatorische Führungskraft eines Softwareentwicklungsteams im JEE Umfeld. Verantwortlich für Entwicklung und Betrieb des 1\&1 VoIP Rufnummernverwaltungssystems und des 1\&1 AutoConfigurationServer.
}

\cventry{04/05--02/08}{Teamleiter Pustefix} {1\&1 Internet AG}{Karlsruhe}{Festanstellung}{
Fachliche, disziplinarische und organisatorische Führungskraft eines Softwareentwicklungsteams im JEE Umfeld. Verantwortlich für die Weiterentwicklung des hausinternen Web-Frameworks Pustefix und zugehöriger Werkzeuge.
}

\cventry{06/02--04/05}{Softwareentwickler} {1\&1 Internet AG}{Karlsruhe}{Festanstellung} {
Durchführung von Projekten im Bereich Webanwendungen für die Bestell- und Endkundenkonfigurationssysteme der 1und1 sowie Weiterentwicklung des hausinternen Webapplikationsframeworks Pustefix.
}

\cventry{1995-2002} {Studium Informatik} {FH Karlsruhe} {}{} {
Studium und Abschluss als Diplom-Informatiker (FH).
} 

\section{Weiterbildung}
\cventry{2017}{Professional Scrum Master II}{scrum.org}{}{}{}
\cventry{2016}{Kanban Management Professional I}{it-agile}{Hamburg}{} {}
\cventry{2015}{Certified LeSS Practitioner}{Craig Larman}{Berlin}{} {}
\cventry{2015}{Certified SAFe Agilist}{wibas GmbH}{Darmstadt}{}{}
\cventry{2015}{Professional Scrum Product Owner I}{scrum.org}{}{}{}
\cventry{2014}{Professional Scrum Master I}{scrum.org}{}{}{}
\cventry{2011}{Theorie und Anwendung von Persönlichkeitsmodellen}{Dart Consulting}{Karlsruhe}{}{}
\cventry{2008--2014}{Leadership Development Program 3}{1\&1 Internet AG}{Karlsruhe}{}{}

\section{Kenntnisse}

\cvitem{Agile}{}
\cvitem{}{%
  \begin{tabular}{p{5cm}l}
  Scrum Master        & ++++ \\
  Product Owner       & +++  \\
  Kanban              & +++   \\
  \end{tabular}%
}

\cvitem{Management}{}
\cvitem{}{%
  \begin{tabular}{p{5cm}l}
  Teamleitung                  & ++++ \\
  Personalauswahl              & ++++ \\
  Personalentwicklung          & ++++  \\
  Resourcen- u. Terminplanung  & ++++ \\
  Strategieentwicklung         & +++   \\
  \end{tabular}%
}

\cvitem{Technik}{}
\cvitem{}{%
  \begin{tabular}{p{5cm}l}
  JEE Technologien             & +++ \\
  DevOps                       & +++ \\
  Continous Integration        & +++  \\
  BPMN                         & +++  \\
  Datenbanken                  & +++ \\
  JIRA                         & +++   \\
  Hochlastsysteme              & +++ \\
  \end{tabular}%
}
\cvitem{    }{}

\cvitem{+++++}{Exzellente Kenntnisse. Löst aktiv schwierigste Fragestellungen.}
\cvitem{++++}{Fortgeschrittene Kenntnisse. Löst aktiv schwierige Fragestellungen.}
\cvitem{+++}{Gute Mittelstufe. Setzt Aufgaben im Fachgebiet selbstständig um.}
\cvitem{++}{Mittelstufe. Setzt Routineaufgaben selbstständig ohne Anleitung um.}
\cvitem{+}{Grundlagen. Kann einfache Routineaufgaben unter Anleitung umsetzen.}
\cvitem{o}{Einstieg. Kann Anwendung und Hintergründe grob erläutern.}



\end{document}
